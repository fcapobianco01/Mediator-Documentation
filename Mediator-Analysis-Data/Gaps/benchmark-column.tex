\begin{table*}
    \centering
    \begin{tabular}{r||ccc||c||cccccc|c|ccc|}
    
    &
    \rotfortyfive{total hooks} &
    \rotfortyfive{hooks analyzed} &
    \rotfortyfive{no sink function} &
    \rotfortyfive{case I violations} &
    \rotfortyfive{sub $\rightarrow$ obj} &
    \rotfortyfive{sub $\rightarrow$ op}  &
    \rotfortyfive{obj $\rightarrow$ sub} &
    \rotfortyfive{obj $\rightarrow$ op}  &
    \rotfortyfive{op  $\rightarrow$ sub} &
    \rotfortyfive{op  $\rightarrow$ obj} &
    \rotfortyfive{dynamic $\rightarrow$ static} &
    \rotfortyfive{input $\rightarrow$ mediator} &
    \rotfortyfive{external $\rightarrow$ input} &
    \rotfortyfive{external $\rightarrow$ mediator} \\ \hline
    
    
SELinux    & 173 & 102 &  71 &   / & 109 &   0 &  58 &  13 & 323 & 404 &  40 & 432 &   0 & 903 \\
Tomoyo     &  28 &  23 &   5 &  20 &   0 &   0 &   8 &  16 &   3 &  20 &  43 &  91 &   0 &  48 \\
AppArmor   &  34 &  20 &  14 &  10 &   0 &   0 &  18 &   2 &   7 &   9 &   9 &  66 &   0 &  80 \\ \hline
    \end{tabular}
    \caption{The number of violations observed in each LSM after enabling relaxed noninterference. \\ \footnotesize{\textit{This includes all violations from explicit flows only and with implicit flows enabled. All such violations are subject to be endorsed by either a simple bypass endorsement, by an inserted endorser in the source code, or be determined to be problematic and should not be endorsed. For each source $x$ with a mapping in $\source(x)$ (an identified source and its source integrity label), and an argument of a sink call and its corresponding expected integrity label ($\policy(s, i)$}), we identify an integrity gap if there exists some information flow (i.e., a path from the source $x$ to a sink calls argument) while $\source(x) \LEQ \policy(s, i)$. (Explain the gaps on DIMENSIONS. Just need the implicit graph.)}} 
    \label{tab:table-lsm-and-gap-flows}
    \end{table*}